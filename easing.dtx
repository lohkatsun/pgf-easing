% \iffalse meta-comment
%
% Copyright (C) 2021 Lo̍h Ka-tsùn
%
% This file may be distributed and/or modified under the conditions of
% the LaTeX Project Public License, either version 1.3 of this license
% or (at your option) any later version.  The latest version of this
% license is in:
%
%     http://www.latex-project.org/lppl.txt
%
% and version 1.3 or later is part of all distributions of LaTeX version
% 2005/12/01 or later.
%
% \fi
% \iffalse
%<*driver>
\documentclass{ltxdoc}
\usepackage{parskip}
\usepackage{pgf}
\usepgflibrary{easing}
\EnableCrossrefs
\CodelineIndex
\RecordChanges
\begin{document}
\DocInput{easing.dtx}
\end{document}
%</driver>
% \fi
% \title{The \textsf{easing} Library for \textsf{pgfmath}}
% \author{Loh Ka-tsun}
% \maketitle
% \section{Usage}
%
% \StopEventually{}
% \section{Implementation}

% \begin{macro}{\ifeasing@withfpu}
% \begin{macro}{\easing@divide}

% This library uses \TeX{} registers and |pgf|'s mathematical engine for
% computations.

% It is possible that the user is loading this library together with
% |fpu|.  We save the basic routines from |pgfmath| so that when this
% happens, |fpu| doesn't break everything when it does a switcharoo
% with the |pgfmath| macros.

%    \begin{macrocode}
\newif\ifeasing@withfpu
\expandafter\ifx\csname pgflibraryfpuifactive\endcsname\relax
\easing@withfpufalse
\else
\easing@withfputrue
\fi
\ifeasing@withfpu
\let\easing@divide\pgfmath@basic@divide@
\let\easing@cos\pgfmath@basic@cos@
\let\easing@exp\pgfmath@basic@exp@
\let\easing@ln\pgfmath@basic@ln@
\else
\let\easing@divide\pgfmathdivide@
\let\easing@cos\pgfmathcos@
\let\easing@exp\pgfmathexp@
\let\easing@ln\pgfmathln@
\fi
%    \end{macrocode}
% \end{macro}
% \end{macro}

% \begin{macro}{\easing@linearstep@ne}
% \begin{macro}{\easing@linearstep@fixed}
% \begin{macro}{\easing@linearstep@float}
% \begin{macro}{\easing@linearstep}

% In absence of |fpu|, the next section of code defines
% |\easing@linearstep|, which expects as arguments plain numbers
% (i.e. things that can be assigned to dimension registers). The net
% effect of |\easing@linearstep{#1}{#2}{#3}| is to set
% |\pgfmathresult| to
% $\frac{\mathtt{\#3}-\mathtt{\#1}}{\mathtt{\#2}-\mathtt{\#1}}$,
% clamped to between $0$ and $1$.

% If |fpu| is loaded, |\easing@linearstep| is instead named
% |\easing@linearstep@fixed|, and we additionally define
% |\easing@linearstep@float|, which expects |fpu|-format floats as
% arguments. We do not format the output as a float since |fpu| is
% smart enough to do that conversion quietly on its own.

% The |\easing@linearstep| routine is the first step in the definition
% of all other routines that compute easing functions.

%    \begin{macrocode}
\def\easing@linearstep@ne#1{%
  \begingroup
  \pgf@x#1pt
  \ifdim1pt<\pgf@x\pgf@x 1pt\fi
  \ifdim0pt>\pgf@x\pgf@x 0pt\fi
  \pgfmathreturn\pgf@x
  \endgroup
}%
\expandafter\def
\csname easing@linearstep\ifeasing@withfpu @fixed\fi\endcsname#1#2#3{%
  \begingroup
  \pgf@xa#3pt
  \pgf@xb#2pt
  \pgf@xc#1pt
  \ifdim\pgf@xb=\pgf@xc
  \edef\pgfmathresult{\ifdim\pgf@xa>\pgf@xb 1\else 0\fi}%
  \else
  \advance\pgf@xa-\pgf@xc
  \advance\pgf@xb-\pgf@xc
  \easing@divide{\pgfmath@tonumber\pgf@xa}{\pgfmath@tonumber\pgf@xb}%
  \easing@linearstep@ne\pgfmathresult 
  \fi
  \pgfmathsmuggle\pgfmathresult
  \endgroup
}%
\ifeasing@withfpu
\def\easing@linearstep@float#1#2#3{%
  \begingroup
  \pgfmathfloatsubtract{#3}{#1}%
  \edef\pgf@tempa{\pgfmathresult}%
  \pgfmathfloatsubtract{#2}{#1}%
  \edef\pgf@tempb{\pgfmathresult}%
  \pgfmathfloatifflags{\pgf@tempb}{0}{%
    \pgfmathfloatifflags{\pgf@tempa}{-}{%
      \edef\pgfmathresult{0}%
    }{%
      \edef\pgfmathresult{1}%
    }%
  }{%
    \pgfmathfloatdivide\pgf@tempa\pgf@tempb
    \pgfmathfloattofixed{\pgfmathresult}%
    \easing@linearstep@ne\pgfmathresult
  }%
  \pgfmathsmuggle\pgfmathresult
  \endgroup
}%
\def\easing@linearstep#1#2#3{%
  \pgflibraryfpuifactive{%
    \easing@linearstep@float{#1}{#2}{#3}}{%
    \easing@linearstep@fixed{#1}{#2}{#3}}%
}%
\fi
%    \end{macrocode}
% \end{macro}
% \end{macro}
% \end{macro}
% \end{macro}

% \begin{macro}{\easing@linearstep@easein@ne}
% \begin{macro}{\easing@linearstep@easeout@ne}

% The linear ease-in and ease-out functions are identitcal to the
% linear step function. We define the respective macros so as not to
% surprise the user with their absence.

%    \begin{macrocode}
\let\easing@lineareasein\easing@linearstep
\pgfmathdeclarefunction{lineareasein}{3}{%
  \easing@lineareasein{#1}{#2}{#3}}%
\let\easing@lineareaseout\easing@linearstep
\pgfmathdeclarefunction{lineareaseout}{3}{%
  \easing@lineareasein{#1}{#2}{#3}}%
%    \end{macrocode}
% \end{macro}
% \end{macro}

% ^^A the next few macros have long names, so we scoot this page to
%     the right to make space in the margins:

% \begingroup
% \addtolength\oddsidemargin{50pt}
% \addtolength\evensidemargin{50pt}

% \begin{macro}{\easing@derive@easein@nefromstep@ne}
% \begin{macro}{\easing@derive@easeout@nefromstep@ne}
% \begin{macro}{\easing@derive@step@nefromeasein@ne}
% \begin{macro}{\easing@derive@easeout@nefromeasein@ne}

% The pattern in general is that, for each shape, we define the
% one-parameter version of the step, ease-in, and ease-out routines
% interpolating between values $0$ at $1$ at the ends of the unit
% interval. Then by composing with |\easing@linearstep|, we obtain the
% three-parameter versions that allow the user to specify the begin
% and end points of the interpolation.

% Most of the time it suffices to define just one of the three
% one-parameter versions of a shape to be able to infer the form of
% all three. This is done with the  |\easing@derive|--|from|-- macros.

%    \begin{macrocode}
\def\easing@derive@easein@nefromstep@ne#1{%
  \expandafter\def\csname easing@#1easein@ne\endcsname##1{%
    \begingroup
    \pgf@x##1 pt
    \divide\pgf@x 2
    \csname easing@#1step@ne\endcsname{\pgfmath@tonumber\pgf@x}%
    \pgf@x\pgfmathresult pt
    \multiply\pgf@x 2
    \pgfmathreturn\pgf@x
    \endgroup
  }%
}%
\def\easing@derive@easeout@nefromstep@ne#1{%
  \expandafter\def\csname easing@#1easeout@ne\endcsname##1{%
    \begingroup
    \pgf@x##1 pt
    \divide\pgf@x 2
    \advance\pgf@x 0.5pt
    \csname easing@#1step@ne\endcsname{\pgfmath@tonumber\pgf@x}%
    \pgf@x\pgfmathresult pt
    \multiply\pgf@x 2
    \advance\pgf@x -1pt
    \pgfmathreturn\pgf@x
    \endgroup
  }%
}%
\def\easing@derive@step@nefromeasein@ne#1{%
  \expandafter\def\csname easing@#1step@ne\endcsname##1{%
  \begingroup
    \pgf@x##1 pt
    \multiply\pgf@x 2
    \ifdim\pgf@x<1pt
    \csname easing@#1easein@ne\endcsname{\pgfmath@tonumber\pgf@x}%
    \pgf@x\pgfmathresult pt
    \divide\pgf@x 2
    \else
    \multiply\pgf@x -1
    \advance\pgf@x 2pt
    \csname easing@#1easein@ne\endcsname{\pgfmath@tonumber\pgf@x}%
    \pgf@x\pgfmathresult pt
    \divide\pgf@x 2
    \multiply\pgf@x -1
    \advance\pgf@x 1pt
    \fi
    \pgfmathreturn\pgf@x
    \endgroup
  }%
}%
\def\easing@derive@easeout@nefromeasein@ne#1{%
  \expandafter\def\csname easing@#1easeout@ne\endcsname##1{%
    \begingroup
    \pgf@x##1pt
    \multiply\pgf@x -1
    \advance\pgf@x 1pt
    \csname easing@#1easein@ne\endcsname{\pgfmath@tonumber\pgf@x}%
    \pgf@x\pgfmathresult pt
    \multiply\pgf@x -1
    \advance\pgf@x 1pt
    \pgfmathreturn\pgf@x
    \endgroup
  }%
}
%    \end{macrocode}
% \end{macro}
% \end{macro}
% \end{macro}
% \end{macro}
% \endgroup ^^A marginpar modifications

% \begin{macro}{\easing@pgfmathinstall}

% The three-parameter versions of each routine is installed
% into the mathematical engine, so that they are available in
% |\pgfmathparse|.

%    \begin{macrocode}
\def\easing@pgfmathinstall#1{%
  \pgfmathdeclarefunction{#1step}{3}{%
    \easing@linearstep{##1}{##2}{##3}%
    \csname easing@#1step@ne\endcsname\pgfmathresult
  }%
  \pgfmathdeclarefunction{#1easein}{3}{%
    \easing@linearstep{##1}{##2}{##3}%
    \csname easing@#1easein@ne\endcsname\pgfmathresult
  }%
  \pgfmathdeclarefunction{#1easeout}{3}{%
    \easing@linearstep{##1}{##2}{##3}%
    \csname easing@#1easeout@ne\endcsname\pgfmathresult
  }%
}%
%    \end{macrocode}
% \end{macro}

% \begin{macro}{\easing@smoothstep@ne}
% \begin{macro}{\easing@smootheasein@ne}
% \begin{macro}{\easing@smootheaseout@ne}

% The |smooth| shape.

%    \begin{macrocode}
\def\easing@smoothstep@ne#1{%
  \begingroup
  \pgf@x#1pt
  \edef\pgf@temp{\pgfmath@tonumber\pgf@x}%
  \multiply\pgf@x-2
  \advance\pgf@x 3pt
  \pgf@x\pgf@temp\pgf@x
  \pgf@x\pgf@temp\pgf@x
  \pgfmathreturn\pgf@x
  \endgroup
}%
\easing@derive@easein@nefromstep@ne{smooth}%
\easing@derive@easeout@nefromstep@ne{smooth}%
\easing@pgfmathinstall{smooth}%
%    \end{macrocode}
% \end{macro}
% \end{macro}
% \end{macro}

% \begin{macro}{\easing@sinestep@ne}
% \begin{macro}{\easing@sineeasein@ne}
% \begin{macro}{\easing@sineeaseout@ne}

% The |sine| shape.

% We write down both the |easein| and |step| forms of this, since they
% are simple compared to what would have been obtained by
% |\easing@derive|--.

%    \begin{macrocode}
\def\easing@sineeasein@ne#1{%
  \begingroup
  \pgf@x#1pt
  \multiply\pgf@x 90
  \easing@cos{\pgfmath@tonumber\pgf@x}%
  \pgf@x\pgfmathresult pt
  \multiply\pgf@x -1
  \advance\pgf@x 1pt
  \pgfmathreturn\pgf@x
  \endgroup
}%
\def\easing@sinestep@ne#1{%
  \begingroup
  \pgf@x#1pt
  \multiply\pgf@x 180
  \easing@cos{\pgfmath@tonumber\pgf@x}%
  \pgf@x\pgfmathresult pt
  \divide\pgf@x 2
  \multiply\pgf@x -1
  \advance\pgf@x 0.5pt
  \pgfmathreturn\pgf@x
  \endgroup
}%
\easing@derive@easeout@nefromeasein@ne{sine}%
\easing@pgfmathinstall{sine}%
%    \end{macrocode}
% \end{macro}
% \end{macro}
% \end{macro}

% \begin{macro}{\easing@powstep@ne}
% \begin{macro}{\easing@poweasein@ne}
% \begin{macro}{\easing@poweaseout@ne}

% The |pow| shape.

% Because of some wonkiness in in |fpu|, instead of invoking the |pow|
% function from |pgfmath|, we compute $t^n$ approximately by computing
% $e^{n\ln t}$ using |ln| and |exp| instead (which is what |pgfmath|
% does anyway when the exponent is not an integer.)

%    \begin{macrocode}
\pgfkeys{/easing/.is family}%
\pgfkeys{easing,
  pow/exponent/.estore in=\easing@param@pow@exponent,
  pow/exponent/.default=2.4,
  pow/exponent}%
\def\easing@poweasein@ne#1{%
  \begingroup
  \pgf@x#1pt
  \ifdim\pgf@x=0pt
  \edef\pgfmathresult{0}%
  \else
  \easing@ln{#1}%
  \pgf@x\pgfmathresult pt
  \pgf@x\easing@param@pow@exponent\pgf@x
  \easing@exp{\pgfmath@tonumber\pgf@x}%
  \fi
  \pgfmathsmuggle\pgfmathresult
  \endgroup
}%
\easing@derive@easeout@nefromeasein@ne{pow}%
\easing@derive@step@nefromeasein@ne{pow}%
\easing@pgfmathinstall{pow}%
%    \end{macrocode}
% \end{macro}
% \end{macro}
% \end{macro}

% \begin{macro}{\easing@quadstep@ne}
% \begin{macro}{\easing@quadeasein@ne}
% \begin{macro}{\easing@quadeaseout@ne}

% \begin{macro}{\easing@cubicstep@ne}
% \begin{macro}{\easing@cubiceasein@ne}
% \begin{macro}{\easing@cubiceaseout@ne}

% \begin{macro}{\easing@quartstep@ne}
% \begin{macro}{\easing@quarteasein@ne}
% \begin{macro}{\easing@quarteaseout@ne}

% \begin{macro}{\easing@quintstep@ne}
% \begin{macro}{\easing@quinteasein@ne}
% \begin{macro}{\easing@quinteaseout@ne}

% The |quad|--, |cubic|--, |quart|--, and |quint|-- routines have
% explicit definitions. The small integer exponents are computed with
% \TeX{} registers, which is probably a little faster and more
% accurate than setting the argument then evaluating the equivalent
% |pow|-- routine.

%    \begin{macrocode}
\def\easing@quadeasein@ne#1{%
  \begingroup
  \pgf@x#1pt
  \edef\pgf@temp{\pgfmath@tonumber\pgf@x}%
  \pgf@x\pgf@temp\pgf@x
  \pgfmathreturn\pgf@x
  \endgroup
}%
\easing@derive@step@nefromeasein@ne{quad}%
\easing@derive@easeout@nefromeasein@ne{quad}%
\easing@pgfmathinstall{quad}%

\def\easing@cubiceasein@ne#1{%
  \begingroup
  \pgf@x#1pt
  \edef\pgf@temp{\pgfmath@tonumber\pgf@x}%
  \pgf@x\pgf@temp\pgf@x
  \pgf@x\pgf@temp\pgf@x
  \pgfmathreturn\pgf@x
  \endgroup
}%
\easing@derive@step@nefromeasein@ne{cubic}%
\easing@derive@easeout@nefromeasein@ne{cubic}%
\easing@pgfmathinstall{cubic}%

\def\easing@quarteasein@ne#1{%
  \begingroup
  \pgf@x#1pt
  \edef\pgf@temp{\pgfmath@tonumber\pgf@x}%
  \pgf@x\pgf@temp\pgf@x
  \pgf@x\pgf@temp\pgf@x
  \pgf@x\pgf@temp\pgf@x
  \pgfmathreturn\pgf@x
  \endgroup
}%
\easing@derive@step@nefromeasein@ne{quart}%
\easing@derive@easeout@nefromeasein@ne{quart}%
\easing@pgfmathinstall{quart}%

\def\easing@quinteasein@ne#1{%
  \begingroup
  \pgf@x#1pt
  \edef\pgf@temp{\pgfmath@tonumber\pgf@x}%
  \pgf@x\pgf@temp\pgf@x
  \pgf@x\pgf@temp\pgf@x
  \pgf@x\pgf@temp\pgf@x
  \pgf@x\pgf@temp\pgf@x
  \pgfmathreturn\pgf@x
  \endgroup
}%
\easing@derive@step@nefromeasein@ne{quint}%
\easing@derive@easeout@nefromeasein@ne{quint}%
\easing@pgfmathinstall{quint}%
%    \end{macrocode}
% \end{macro}
% \end{macro}
% \end{macro}

% \end{macro}
% \end{macro}
% \end{macro}

% \end{macro}
% \end{macro}
% \end{macro}

% \end{macro}
% \end{macro}
% \end{macro}

% \begin{macro}{\easing@backstep@ne}
% \begin{macro}{\easing@backeasein@ne}
% \begin{macro}{\easing@backeaseout@ne}

% The |back| shape.

%    \begin{macrocode}
\pgfkeys{easing,
  back/overshoot/.estore in=\easing@param@back@overshoot,
  back/overshoot/.default=1.6,
  back/overshoot}%
\def\easing@backeasein@ne#1{%
  \begingroup
  \pgf@x#1pt
  \edef\pgf@temp{\pgfmath@tonumber\pgf@x}%
  \advance\pgf@x -1pt
  \pgf@x\easing@param@back@overshoot\pgf@x
  \advance\pgf@x\pgf@temp pt
  \pgf@x\pgf@temp\pgf@x
  \pgf@x\pgf@temp\pgf@x
  \pgfmathreturn\pgf@x  
  \endgroup
}%
\easing@derive@step@nefromeasein@ne{back}%
\easing@derive@easeout@nefromeasein@ne{back}%
\easing@pgfmathinstall{back}%
%    \end{macrocode}
% \end{macro}
% \end{macro}
% \end{macro}


% \Finale

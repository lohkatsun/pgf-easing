% \iffalse meta-comment
%
% Copyright (C) 2021 Lo̍h Ka-tsùn
%
% This file may be distributed and/or modified under the conditions of
% the LaTeX Project Public License, either version 1.3 of this license
% or (at your option) any later version.  The latest version of this
% license is in:
%
%     http://www.latex-project.org/lppl.txt
%
% and version 1.3 or later is part of all distributions of LaTeX version
% 2005/12/01 or later.
%
% \fi
% \iffalse
%<*driver>
\documentclass{ltxdoc}
\usepackage{parskip}
\usepackage{pgfplots}
\usepgflibrary{easing}
% \usepackage{hyperref}
\newcommand{\pgf}{\textsc{pgf}}
\EnableCrossrefs
\CodelineIndex
\RecordChanges
\CheckSum{594}
\changes{0.1}{2021/07/18}{Initial version}
\begin{document}
\DocInput{easing.dtx}
\end{document}
%</driver>
% \fi
% ^^A for drawing nice little graphs of the functions. pgfplots is
% ^^A probably overkill for this, so here's some ad-hoc TikZ-ing;
% ^^A hopefully to be tidied up substantially in a future version:
%
%   \def\graphmargindefaulty{0.16}
%   \def\graphmarginx{0.12}
%   \edef\graphmarginy{\graphmargindefaulty}
% \newenvironment{easinggraph}{%
% \begin{tikzpicture}[x=2.8cm,y=2.8cm]
%   \fill[white!97!black, use as bounding box]
%     (-\graphmarginx,-\graphmarginy) rectangle
%     ({1+\graphmarginx},{1+\graphmarginy});
%   \draw[white!60!black]
%     ({1+\graphmarginx},0) -- (-\graphmarginx,0) node[left] {$0$}
%     ({1+\graphmarginx},1) -- (-\graphmarginx,1) node[left] {$1$};
%   \draw[white!60!black,dashed]
%     (0,1) -- (0,-\graphmarginy) node[below] {$a$}
%     (1,1) -- (1,-\graphmarginy) node[below] {$b$};
% }{\end{tikzpicture}}
% \newcommand{\easinggraphplot}[2]{%
%   \pgfmathparse{#2}
%   \draw[white!\pgfmathresult!black,thick]
%     (-\graphmarginx,0) -- (0,0)
%     \foreach\x in{0,0.01,...,1}{--(\x,{#1(0,1,\x)})}
%     -- (1,1) -- ({1+\graphmarginx},1);
% }
% \newcommand{\easinggraphplotfive}[4]{%
%   \easinggraphlabel{#1\easingtype}{
%     \foreach[count=\countv]\paramv
%       in{#4}{\pgfmathparse{
%         18*(5-\countv)}{\color{
%         white!\pgfmathresult!black}\ifnum\countv=1\relax
%         #3=\else,\fi\paramv}}}
%   \foreach[count=\countv]\paramv in{#4}{%
%     \pgfkeys{/easing/#1/#2=\paramv}
%     \easinggraphplot{#1\easingtype}{18*(5-\countv)}
%   }
% }
% \newcommand{\easinggraphlabel}[2]{%
%   \node[anchor=text] at (-\graphmarginx,{1+\graphmarginy})
%     {\footnotesize\tt\raisebox{\baselineskip}{\parbox{3.5cm}{#1\\[-0.4ex]#2}}};
% }
% \newcommand{\foreachtypeeasinggraph}[1]{%
%   \vspace{2\baselineskip}
%   \foreach[count=\easingtypen]\easingtype in{easein,step,easeout}{%
%   \ifnum\easingtypen>1\hspace{\fill}\fi
%   \begin{easinggraph}#1\end{easinggraph}
%   }
%   \vspace{\baselineskip}
% }
%
% ^^A for printing code listings. again, there's probably a "proper"
% ^^A way to do this, but this is good enough i guess
%
% \newenvironment{fakelisting}{
%   \begingroup\parskip 0pt\par\hrulefill}{
%   \vspace{-1.3\baselineskip}\hrulefill\endgroup
%   \vspace{\parskip}}
%
% ^^A body of documentation proper begins here.
%
% \title{The \textsf{easing}\thanks{version 0.1, dated
% 2021/07/18. This version was tested with \pgf{} version 3.1.9a}
% \ Library for \pgf}
%
% \author{Loh Ka-tsun\thanks{\texttt{loh.ka.tsun@gmail.com}}}
% \date{}
% \maketitle
%
% \thispagestyle{empty}
% \vspace{-2em}
% \begin{center}
% \begin{easinggraph}
% \easinggraphplot{smootherstep}{46}
% \easinggraphplot{smoothstep}{0}
% \end{easinggraph}
% \end{center}
%
% \tableofcontents\pagebreak
%
% \section{Introduction}
% This library adds easing functions to the \pgf{} mathematical engine.
%
% \section{Installation}
%
% The \textsf{easing} library is a \pgf{} library; it works both with
% \LaTeX{} and with plain \TeX{}. Once the file
% |pgflibraryeasing.code.tex| is in a directory searched by \TeX, the
% library can be loaded as follows:
%
% with plain \TeX:
% \begin{fakelisting}\begin{verbatim}
% \input pgf
% \usepgflibrary{easing}
% \end{verbatim}\end{fakelisting}
%
% with \LaTeX:
% \begin{fakelisting}\begin{verbatim}
% \usepackage{pgf}
% \usepgflibrary{easing}
% \end{verbatim}\end{fakelisting}
%
% The \textsf{easing} library is compatible with, but does not depend
% on, the floating point unit library provided by \pgf{}. To use both
% \textsf{easing} and the FPU, the FPU (or any packages/libraries
% which use the FPU, such as \textsf{pgfplots}) must be loaded
% before the \textsf{easing} library.
%
% \section{Usage}
%
% The routines implemented by the \textsf{easing} library are added to
% \pgf{}'s mathematical engine with |\pgfmathdeclarefunction|, so that
% they are recognised by by |\pgfmathparse| and can be used in any
% expression which is processed by the parser.
% As a first example, the following code produces plots of the
% function |smoothstep(|$a$|,|$b$|,|$x$|)| against the argument $x$,
% with one endpoint $a=0$ and the other endpoint $b$ ranging through
% the integers $-1$ to $3$:
%
% \vspace{\parskip}
% \hspace{-0.05\textwidth}
% \begin{minipage}[h]{0.55\textwidth}
% \tikzpicture
% \axis[
%   width=\textwidth, ^^A we lie a little and omit this from the source listing
%   domain=-1.2:3.2, samples=64,
%   xmin=-1.2, xmax=3.2,
%   cycle list={[samples of colormap=6 of viridis]},
%   no marks, thick]
% \pgfplotsinvokeforeach{-1,...,3}{\addplot{smoothstep(0,#1,x)};}
% \endaxis
% \endtikzpicture
% \end{minipage}%
% \begin{minipage}[h]{0.5\textwidth}
% \begin{fakelisting}\begin{verbatim}
% \input pgfplots
% \usepgflibrary{easing}
% \tikzpicture
% \axis[
%   domain=-1.2:3.2, samples=64,
%   xmin=-1.2, xmax=3.2,
%   cycle list={
%     [samples of colormap=6 of viridis]},
%   no marks, thick]
% \pgfplotsinvokeforeach{-1,...,3}{
%   \addplot{smoothstep(0,#1,x)};}
% \endaxis
% \endtikzpicture
% \end
% \end{verbatim}\end{fakelisting}
% \end{minipage}
% \vspace{\parskip}
%
% (This example also demonstrates the behaviour of the easing
% functions in some special cases: when the endpoints $b\le a$, and in
% particular the degenerate case where $a=b$, in which the library
% chooses to consider the function that is $1$ for all $x\ge 0$ and
% $0$ otherwise.)
%
% Like all functions declared in this way, the functions implemented
% by \textsf{easing} are also available as ``public'' macros, such as
% |\pgfmathsmoothstep|:
%
% \vspace{\parskip}
% \begin{minipage}[h]{0.5\textwidth}
% \foreach\x in{0,0.25,...,1}{
%   \pgfmathsmoothstep{0}{1}{\x}
%   $S_1(\x)=\pgfmathresult$\par
% }
% \end{minipage}%
% \begin{minipage}[h]{0.5\textwidth}
% \begin{fakelisting}\begin{verbatim}
% \input pgf
% \usepgflibrary{easing}
% \foreach\x in{0,0.25,...,1}{
%   \pgfmathsmoothstep{0}{1}{\x}
%   $S_1(\x)=\pgfmathresult$\par
% }
% \end
% \end{verbatim}\end{fakelisting}
% \end{minipage}
% \vspace{\parskip}
%
% See Part VIII of the \pgf{} manual for more details on the
% mathematical engine.
%
% \subsection{Naming conventions}
%
% For each shape, three functions are declared, all of which take
% three arguments $a,b$, and $x$. Where $a<b$, all of these function
% take value $0$ whenever $x\le a$ and $1$ whenever $x\ge b$. The
% names of the functions adhere to the following pattern:
%
% \begin{itemize}
%
% \item The \emph{ease-in} form \meta{shape}|easein(|$a$|,|$b$|,|$x$|)| has easing
% applied near the endpoint $a$.
%
% \item The \emph{ease-out} form \meta{shape}|easeout(|$a$|,|$b$|,|$x$|)| has
% easing applied near the endpoint $b$. Its graph is that of the
% ease-in form reflected about both axes.
%
% \item The \emph{step function} form \meta{shape}|step(|$a$|,|$b$|,|$x$|)| has
% easing applied near both endpoints. Its graph is that of the ease-in
% and ease-out forms concatenated then appropriately scaled.
%
% \end{itemize}
%
% \vspace{\parskip}
% \begin{minipage}[h]{0.5\textwidth}
% \tikzpicture
% \foreach\x in{0,...,12}{
%   \draw[gray,dashed]
%     (0,-1) -- (0,4) (5,-1) -- (5,4);
%   \draw[thick]
%     ({5*smootheasein(0,12,\x)},3)
%     circle (0.25)
%     ({5*smoothstep(0,12,\x)},1.5)
%     circle (0.25)
%     ({5*smootheaseout(0,12,\x)},0)
%     circle (0.25);
% }
% \endtikzpicture
% \end{minipage}%
% \begin{minipage}[h]{0.5\textwidth}
% \begin{fakelisting}\begin{verbatim}
% \input tikz
% \usepgflibrary{easing}
% \tikzpicture
% \foreach\x in{0,...,12}{
%   \draw[gray,dashed]
%     (0,-1) -- (0,4) (5,-1) -- (5,4);
%   \draw[thick]
%     ({5*smootheasein(0,12,\x)},3)
%     circle (0.25)
%     ({5*smoothstep(0,12,\x)},1.5)
%     circle (0.25)
%     ({5*smootheaseout(0,12,\x)},0)
%     circle (0.25);
% }
% \endtikzpicture
% \end
% \end{verbatim}\end{fakelisting}
% \end{minipage}
% \vspace{\parskip}
%
% \subsection{Specifying parameters}
%
% Some of these shapes can be modified by adjusting one or more
% parameters, which is done through |pgfkeys|: the parameter
% \meta{param} for functions of shape \meta{shape} is specified by
% setting the \pgf{} key |/easing/|\meta{shape}|/|\meta{param}:
%
% \vspace{\parskip}
% \begin{minipage}[h]{0.5\textwidth}
% \hspace{-0.11\textwidth}
% \tikzpicture
% \axis[
%   width=1.1\textwidth, ^^A give up on understanding how pgfplots
%                        ^^A computes the width of a graph and just
%                        ^^A eyeball it
%   domain=-0.2:1.2, samples=64,
%   xmin=0, xmax=1, enlarge x limits,
%   cycle list={
%     [samples of colormap=6 of viridis]},
%   no marks, thick]
% \pgfplotsinvokeforeach{0,...,4}{
%   \pgfkeys{easing,back/overshoot=#1}
%   \addplot{backeaseout(0,1,x)};
% }
% \endaxis
% \endtikzpicture
% \end{minipage}%
% \begin{minipage}[h]{0.5\textwidth}
% \begin{fakelisting}\begin{verbatim}
% \input pgfplots
% \usepgflibrary{easing}
% \tikzpicture
% \axis[
%   domain=-0.2:1.2, samples=64,
%   xmin=0, xmax=1, enlarge x limits,
%   cycle list={
%     [samples of colormap=6 of viridis]},
%   no marks, thick]
% \pgfplotsinvokeforeach{0,...,4}{
%   \pgfkeys{easing,back/overshoot=#1}
%   \addplot{backeaseout(0,1,x)};
% }
% \endaxis
% \endtikzpicture
% \end
% \end{verbatim}\end{fakelisting}
% \end{minipage}
% \vspace{\parskip}
%
% Setting a parameter affects the ease-in, step, and ease-out forms of
% the relevant function at once.
%
% For detailed descriptions of the parameters admitted by each shape,
% see the following section.
%
% \section{List of easing function shapes}
%
% An exhaustive list follows of all the easing functions implemented
% by the \textsf{easing} library. For clarity, where mathematical
% expressions are given for functions, they are written in terms of
% a parameter $t$ equal to $\frac{x}{b-a}$.
%
% \subsection{Polynomials}
%
% \subsubsection{The \texttt{smooth} and \texttt{smoother} shapes}
%
% The step function form of the |smooth| shape is a third-order
% Hermite polynomial interpolation between $0$ and $1$, so that the
% first derivate at the endpoints are zero. It is defined $3t^2-2t^3$
% for $0\le t\le 1$.
%
% The step function form of the |smoother| shape is a fifth-order
% Hermite polynomial interpolation between $0$ and $1$, so that the
% first and second derivatives at the endpoints are zero. It is defined
% $10t^3-15t^4+6t^5$ for $0\le t\le 1$.
%
% \foreachtypeeasinggraph{%
%     \easinggraphlabel{smooth\easingtype}{{\color{white!46!black}smoother\easingtype}}
%     \easinggraphplot{smoother\easingtype}{46}
%     \easinggraphplot{smooth\easingtype}{0}
% }
%
% \subsubsection{The \texttt{pow} shape and friends (\texttt{linear}, \texttt{quad},
% \texttt{cubic}, \texttt{quart}, and \texttt{quint})}
%
% Polynomial easing. The ease-in form is defined as $t^n$ for $0\le
% t\le 1$, where the exponent $n$ is set with the \pgf{} key
% |/easing/pow/exponent|, and should be greater than $0$. The parameter
% defaults to $n=2.4$.
%
% When $n=1$, the function is linear between $0$ and $1$. For
% $0<n\le1$, the ease-in form has discontinuous derivative at $0$.
%
% The shapes |linear|, |quad|, |cubic|, |quart|, and |quint| are the
% same functions as |pow| with $n=1,\dots,5$,
% respectively. Computations for these shapes are implemented with
% \TeX{} registers, which is a little faster and more accurate than
% setting the argument then evaluating the equivalent |pow| function.
%
% \foreachtypeeasinggraph{%
%   \easinggraphplotfive{pow}{exponent}{n}{0.5,1,2,3,4}}
%
% \subsubsection{The \texttt{back} shape}
%
% Anticipatory easing. The ease-in form is defined as $t^2(1-t)s+t^3$
% for $0\le t\le 1$, where the parameter $s$ is set with the \pgf{} key
% |/easing/back/overshoot|. The parameter defaults to $s=1.6$.
%
% When $s\le0$, there is no overshoot. When $s=0$, the function is
% equivalent to |pow| with $n=3$.
%
% \def\graphmarginy{0.3}
% \foreachtypeeasinggraph{%
%   \easinggraphplotfive{back}{overshoot}{s}{-0.8,0.0,0.8,1.6,2.4}}
% \edef\graphmarginy{\graphmargindefaulty}
%
% \subsection{Trigonometric and exponential}
%
% \subsubsection{The \texttt{sine} shape}
%
% An easing function that looks like a section of a sinusoid. The ease-out form is
% defined as $\sin(\frac{\pi}{2}t)$ for $0\le t\le 1$.
%
% This shape admits no parameters.
%
% \foreachtypeeasinggraph{%
%   \easinggraphlabel{\phantom{nothing here}}{sine\easingtype}
%   \easinggraphplot{sine\easingtype}{0}}
%
% \subsubsection{The \texttt{exp} shape}
%
% An easing function that looks like an exponential function. The
% ease-in form is defined as $e^{c (t-1)}$ for $0\le t\le 1$, where
% the parameter $c$ is set with the \pgf{} key |/easing/exp/speed|,
% and should be greater than $0$. The parameter defaults to
% $c=7.2$.
%
% Because of the nature of the exponential function, this shape is
% only approximately continuous at the endpoints $a$ and $b$. In
% practice, the discontinuity only becomes noticeable for small
% $c$, around $c\le 4$.
%
% \foreachtypeeasinggraph{%
%   \easinggraphplotfive{exp}{speed}{c}{4,7,10,13,16}}
%
% \subsection{Other}
%
% \subsubsection{The \texttt{circ} shape}
%
% An easing function whose graph is part of an ellipse. This shape
% admits no parameters.
%
% \foreachtypeeasinggraph{%
%   \easinggraphlabel{\phantom{nothing here}}{circ\easingtype}
%   \easinggraphplot{circ\easingtype}{0}}
%
% \subsubsection{The \texttt{elastic} shape}
%
% Easing function that looks like a damped harmonic oscillator. The
% ease-out form is defined as $e^c(t-1)\cos(2\pi f (1-t))$. This shape
% admits two parameters:
%
% \begin{itemize}
%
% \item The frequency $f$ is the number of oscillations between the
% endpoints. It is set with the \pgf{} key
% |/easing/elastic/frequency|, and should be greater than $0$. The
% frequency defaults to $f=3$.
%
% \item The damping coefficient $b$ affects the speed at which the
% amplitude decays. It is set with the \pgf{} key
% |/easing/elastic/damping|, and should be greater than zero. The
% damping coefficient defaults to $b=7.2$.
%
% \end{itemize}
%
% The function overshoots the range $[0,1]$ when $f>0.5$. For $0<f\le
% 1$, this function becomes a family of anticipatory easing curves
% that look slightly different from the |back| shape but are more
% expensive to compute.
%
% ^^A twiddle with the size of the background so that the curves don't
% ^^A overshoot the box
%
% \def\graphmarginy{0.5}
% \foreachtypeeasinggraph{%
%   \easinggraphplotfive{elastic}{frequency}{f}{0.5,1.4,2.3,3.2,4.1}}
% \edef\graphmarginy{\graphmargindefaulty}
%
% \StopEventually{\PrintChanges}
% \section{Implementation}
%
% \begin{macro}{\ifeasing@withfpu}
% \begin{macro}{\easing@divide}
%
% This library uses \TeX{} registers and \pgf's mathematical engine for
% computations.
%
% It is possible that the user is loading this library together with
% the floating point unit library.  We save the basic routines from
% |pgfmath|, so that when this happens, the FPU doesn't break
% everything when it does a switcharoo with the |pgfmath| macros.
%
%    \begin{macrocode}
\newif\ifeasing@withfpu
\expandafter\ifx\csname pgflibraryfpuifactive\endcsname\relax
\easing@withfpufalse
\else
\easing@withfputrue
\fi
\ifeasing@withfpu
\let\easing@cos\pgfmath@basic@cos@
\let\easing@divide\pgfmath@basic@divide@
\let\easing@exp\pgfmath@basic@exp@
\let\easing@ln\pgfmath@basic@ln@
\let\easing@sqrt\pgfmath@basic@sqrt@
\else
\let\easing@cos\pgfmathcos@
\let\easing@divide\pgfmathdivide@
\let\easing@exp\pgfmathexp@
\let\easing@ln\pgfmathln@
\let\easing@sqrt\pgfmathsqrt@
\fi
%    \end{macrocode}
% \end{macro}
% \end{macro}
%
% \begin{macro}{\easing@linearstep@ne}
% \begin{macro}{\easing@linearstep@fixed}
% \begin{macro}{\easing@linearstep@float}
% \begin{macro}{\easing@linearstep}
%
% In absence of the FPU, the next section of code defines
% |\easing@linearstep|, which expects as arguments plain numbers
% (i.e. things that can be assigned to dimension registers). The net
% effect of |\easing@linearstep{#1}{#2}{#3}| is to set
% |\pgfmathresult| to
% $\frac{\mathtt{\#3}-\mathtt{\#1}}{\mathtt{\#2}-\mathtt{\#1}}$,
% clamped to between $0$ and $1$.
%
% If the FPU is loaded, |\easing@linearstep| is instead named
% |\easing@linearstep@fixed|, and we additionally define
% |\easing@linearstep@float|, which expects FPU-format floats as
% arguments. We do not format the output as a float since the FPU is
% smart enough to do that conversion quietly on its own.
%
% The |\easing@linearstep| routine is the first step in the definition
% of all other routines that compute easing functions.
%
%    \begin{macrocode}
\def\easing@linearstep@ne#1{%
  \begingroup
  \pgf@x#1pt
  \ifdim1pt<\pgf@x\pgf@x 1pt\fi
  \ifdim0pt>\pgf@x\pgf@x 0pt\fi
  \pgfmathreturn\pgf@x
  \endgroup
}%
\expandafter\def
\csname easing@linearstep\ifeasing@withfpu @fixed\fi\endcsname#1#2#3{%
  \begingroup
  \pgf@xa#3pt
  \pgf@xb#2pt
  \pgf@xc#1pt
  \ifdim\pgf@xb=\pgf@xc
  \edef\pgfmathresult{\ifdim\pgf@xa>\pgf@xb 1\else 0\fi}%
  \else
  \advance\pgf@xa-\pgf@xc
  \advance\pgf@xb-\pgf@xc
  \easing@divide{\pgfmath@tonumber\pgf@xa}{\pgfmath@tonumber\pgf@xb}%
  \easing@linearstep@ne\pgfmathresult
  \fi
  \pgfmathsmuggle\pgfmathresult
  \endgroup
}%
\ifeasing@withfpu
\def\easing@linearstep@float#1#2#3{%
  \begingroup
  \pgfmathfloatsubtract{#3}{#1}%
  \edef\pgf@tempa{\pgfmathresult}%
  \pgfmathfloatsubtract{#2}{#1}%
  \edef\pgf@tempb{\pgfmathresult}%
  \pgfmathfloatifflags{\pgf@tempb}{0}{%
    \pgfmathfloatifflags{\pgf@tempa}{-}{%
      \edef\pgfmathresult{0}%
    }{%
      \edef\pgfmathresult{1}%
    }%
  }{%
    \pgfmathfloatdivide\pgf@tempa\pgf@tempb
    \pgfmathfloattofixed{\pgfmathresult}%
    \easing@linearstep@ne\pgfmathresult
  }%
  \pgfmathsmuggle\pgfmathresult
  \endgroup
}%
\def\easing@linearstep#1#2#3{%
  \pgflibraryfpuifactive{%
    \easing@linearstep@float{#1}{#2}{#3}}{%
    \easing@linearstep@fixed{#1}{#2}{#3}}%
}%
\fi
%    \end{macrocode}
% \end{macro}
% \end{macro}
% \end{macro}
% \end{macro}
%
% \begin{macro}{\easing@linearstep@easein@ne}
% \begin{macro}{\easing@linearstep@easeout@ne}
%
% The linear ease-in and ease-out functions are identical to the
% linear step function. We define the respective macros so as not to
% surprise the user with their absence.
%
%    \begin{macrocode}
\let\easing@lineareasein\easing@linearstep
\pgfmathdeclarefunction{lineareasein}{3}{%
  \easing@lineareasein{#1}{#2}{#3}}%
\let\easing@lineareaseout\easing@linearstep
\pgfmathdeclarefunction{lineareaseout}{3}{%
  \easing@lineareasein{#1}{#2}{#3}}%
%    \end{macrocode}
% \end{macro}
% \end{macro}
%
% ^^A the next few macros have long names, so we scoot this page to
% ^^A the right to make space in the margins:
%
% \begingroup
% \addtolength\oddsidemargin{50pt}
% \addtolength\evensidemargin{50pt}
%
% \begin{macro}{\easing@derive@easein@nefromstep@ne}
% \begin{macro}{\easing@derive@easeout@nefromstep@ne}
% \begin{macro}{\easing@derive@step@nefromeasein@ne}
% \begin{macro}{\easing@derive@easeout@nefromeasein@ne}
%
% The pattern in general is that, for each shape, we define the
% one-parameter version of the step, ease-in, and ease-out routines
% interpolating between values $0$ at $1$ at the ends of the unit
% interval. Then by composing with |\easing@linearstep|, we obtain the
% three-parameter versions that allow the user to specify the begin
% and end points of the interpolation.
%
% Most of the time it suffices to define just one of the three
% one-parameter versions of a shape to be able to infer the form of
% all three. This is done with the  |\easing@derive|--|from|-- macros.
%
%    \begin{macrocode}
\def\easing@derive@easein@nefromstep@ne#1{%
  \expandafter\def\csname easing@#1easein@ne\endcsname##1{%
    \begingroup
    \pgf@x##1 pt
    \divide\pgf@x 2
    \csname easing@#1step@ne\endcsname{\pgfmath@tonumber\pgf@x}%
    \pgf@x\pgfmathresult pt
    \multiply\pgf@x 2
    \pgfmathreturn\pgf@x
    \endgroup
  }%
}%
\def\easing@derive@easeout@nefromstep@ne#1{%
  \expandafter\def\csname easing@#1easeout@ne\endcsname##1{%
    \begingroup
    \pgf@x##1 pt
    \divide\pgf@x 2
    \advance\pgf@x 0.5pt
    \csname easing@#1step@ne\endcsname{\pgfmath@tonumber\pgf@x}%
    \pgf@x\pgfmathresult pt
    \multiply\pgf@x 2
    \advance\pgf@x -1pt
    \pgfmathreturn\pgf@x
    \endgroup
  }%
}%
\def\easing@derive@step@nefromeasein@ne#1{%
  \expandafter\def\csname easing@#1step@ne\endcsname##1{%
    \begingroup
    \pgf@x##1 pt
    \multiply\pgf@x 2
    \ifdim\pgf@x<1pt
    \csname easing@#1easein@ne\endcsname{\pgfmath@tonumber\pgf@x}%
    \pgf@x\pgfmathresult pt
    \divide\pgf@x 2
    \else
    \multiply\pgf@x -1
    \advance\pgf@x 2pt
    \csname easing@#1easein@ne\endcsname{\pgfmath@tonumber\pgf@x}%
    \pgf@x\pgfmathresult pt
    \divide\pgf@x 2
    \multiply\pgf@x -1
    \advance\pgf@x 1pt
    \fi
    \pgfmathreturn\pgf@x
    \endgroup
  }%
}%
\def\easing@derive@easeout@nefromeasein@ne#1{%
  \expandafter\def\csname easing@#1easeout@ne\endcsname##1{%
    \begingroup
    \pgf@x##1pt
    \multiply\pgf@x -1
    \advance\pgf@x 1pt
    \csname easing@#1easein@ne\endcsname{\pgfmath@tonumber\pgf@x}%
    \pgf@x\pgfmathresult pt
    \multiply\pgf@x -1
    \advance\pgf@x 1pt
    \pgfmathreturn\pgf@x
    \endgroup
  }%
}%
%    \end{macrocode}
% \end{macro}
% \end{macro}
% \end{macro}
% \end{macro}
% \endgroup ^^A marginpar modifications
%
% \begin{macro}{\easing@pgfmathinstall}
%
% The three-parameter versions of each routine is installed
% into the mathematical engine, so that they are available in
% |\pgfmathparse|.
%
%    \begin{macrocode}
\def\easing@pgfmathinstall#1{%
  \pgfmathdeclarefunction{#1step}{3}{%
    \easing@linearstep{##1}{##2}{##3}%
    \csname easing@#1step@ne\endcsname\pgfmathresult
  }%
  \pgfmathdeclarefunction{#1easein}{3}{%
    \easing@linearstep{##1}{##2}{##3}%
    \csname easing@#1easein@ne\endcsname\pgfmathresult
  }%
  \pgfmathdeclarefunction{#1easeout}{3}{%
    \easing@linearstep{##1}{##2}{##3}%
    \csname easing@#1easeout@ne\endcsname\pgfmathresult
  }%
}%
%    \end{macrocode}
% \end{macro}
%
% \begin{macro}{\easing@smoothstep@ne}
% \begin{macro}{\easing@smootheasein@ne}
% \begin{macro}{\easing@smootheaseout@ne}
%
% The |smooth| shape.
%
%    \begin{macrocode}
\def\easing@smoothstep@ne#1{%
  \begingroup
  \pgf@x#1pt
  \edef\pgf@temp{\pgfmath@tonumber\pgf@x}%
  \multiply\pgf@x-2
  \advance\pgf@x 3pt
  \pgf@x\pgf@temp\pgf@x
  \pgf@x\pgf@temp\pgf@x
  \pgfmathreturn\pgf@x
  \endgroup
}%
\easing@derive@easein@nefromstep@ne{smooth}%
\easing@derive@easeout@nefromstep@ne{smooth}%
\easing@pgfmathinstall{smooth}%
%    \end{macrocode}
% \end{macro}
% \end{macro}
% \end{macro}
%
% \begin{macro}{\easing@smootherstep@ne}
% \begin{macro}{\easing@smoothereasein@ne}
% \begin{macro}{\easing@smoothereaseout@ne}
%
% The |smoother| shape.
%
%    \begin{macrocode}
\def\easing@smootherstep@ne#1{%
  \begingroup
  \pgf@x#1pt
  \edef\pgf@temp{\pgfmath@tonumber\pgf@x}%
  \multiply\pgf@x 6
  \advance\pgf@x -15pt
  \pgf@x\pgf@temp\pgf@x
  \advance\pgf@x 10pt
  \pgf@x\pgf@temp\pgf@x
  \pgf@x\pgf@temp\pgf@x
  \pgf@x\pgf@temp\pgf@x
  \pgfmathreturn\pgf@x
  \endgroup
}%
\easing@derive@easein@nefromstep@ne{smoother}%
\easing@derive@easeout@nefromstep@ne{smoother}%
\easing@pgfmathinstall{smoother}%
%    \end{macrocode}
% \end{macro}
% \end{macro}
% \end{macro}
%
% \begin{macro}{\easing@powstep@ne}
% \begin{macro}{\easing@poweasein@ne}
% \begin{macro}{\easing@poweaseout@ne}
%
% The |pow| shape.
%
% Because of some wonkiness in the FPU, |\pgfmath@pow@basic@| actually
% doesn't work. Instead of invoking the |pow| function, we compute
% $t^n$ approximately by computing $e^{n\ln t}$ using |ln| and |exp|
% instead (which is what |pgfmath| does anyway when the exponent is
% not an integer.)
%
%    \begin{macrocode}
\pgfkeys{/easing/.is family}%
\pgfkeys{easing,
  pow/exponent/.estore in=\easing@param@pow@exponent,
  pow/exponent/.default=2.4,
  pow/exponent}%
\def\easing@poweasein@ne#1{%
  \begingroup
  \pgf@x#1pt
  \ifdim\pgf@x=0pt
  \edef\pgfmathresult{0}%
  \else
  \easing@ln{#1}%
  \pgf@x\pgfmathresult pt
  \pgf@x\easing@param@pow@exponent\pgf@x
  \easing@exp{\pgfmath@tonumber\pgf@x}%
  \fi
  \pgfmathsmuggle\pgfmathresult
  \endgroup
}%
\easing@derive@easeout@nefromeasein@ne{pow}%
\easing@derive@step@nefromeasein@ne{pow}%
\easing@pgfmathinstall{pow}%
%    \end{macrocode}
% \end{macro}
% \end{macro}
% \end{macro}
%
% \begin{macro}{\easing@quadstep@ne}
% \begin{macro}{\easing@quadeasein@ne}
% \begin{macro}{\easing@quadeaseout@ne}
%
% \begin{macro}{\easing@cubicstep@ne}
% \begin{macro}{\easing@cubiceasein@ne}
% \begin{macro}{\easing@cubiceaseout@ne}
%
% \begin{macro}{\easing@quartstep@ne}
% \begin{macro}{\easing@quarteasein@ne}
% \begin{macro}{\easing@quarteaseout@ne}
%
% \begin{macro}{\easing@quintstep@ne}
% \begin{macro}{\easing@quinteasein@ne}
% \begin{macro}{\easing@quinteaseout@ne}
%
% The |quad|--, |cubic|--, |quart|--, and |quint|-- routines have
% explicit definitions.
%
%    \begin{macrocode}
\def\easing@quadeasein@ne#1{%
  \begingroup
  \pgf@x#1pt
  \edef\pgf@temp{\pgfmath@tonumber\pgf@x}%
  \pgf@x\pgf@temp\pgf@x
  \pgfmathreturn\pgf@x
  \endgroup
}%
\easing@derive@step@nefromeasein@ne{quad}%
\easing@derive@easeout@nefromeasein@ne{quad}%
\easing@pgfmathinstall{quad}%
%
\def\easing@cubiceasein@ne#1{%
  \begingroup
  \pgf@x#1pt
  \edef\pgf@temp{\pgfmath@tonumber\pgf@x}%
  \pgf@x\pgf@temp\pgf@x
  \pgf@x\pgf@temp\pgf@x
  \pgfmathreturn\pgf@x
  \endgroup
}%
\easing@derive@step@nefromeasein@ne{cubic}%
\easing@derive@easeout@nefromeasein@ne{cubic}%
\easing@pgfmathinstall{cubic}%
%
\def\easing@quarteasein@ne#1{%
  \begingroup
  \pgf@x#1pt
  \edef\pgf@temp{\pgfmath@tonumber\pgf@x}%
  \pgf@x\pgf@temp\pgf@x
  \pgf@x\pgf@temp\pgf@x
  \pgf@x\pgf@temp\pgf@x
  \pgfmathreturn\pgf@x
  \endgroup
}%
\easing@derive@step@nefromeasein@ne{quart}%
\easing@derive@easeout@nefromeasein@ne{quart}%
\easing@pgfmathinstall{quart}%
%
\def\easing@quinteasein@ne#1{%
  \begingroup
  \pgf@x#1pt
  \edef\pgf@temp{\pgfmath@tonumber\pgf@x}%
  \pgf@x\pgf@temp\pgf@x
  \pgf@x\pgf@temp\pgf@x
  \pgf@x\pgf@temp\pgf@x
  \pgf@x\pgf@temp\pgf@x
  \pgfmathreturn\pgf@x
  \endgroup
}%
\easing@derive@step@nefromeasein@ne{quint}%
\easing@derive@easeout@nefromeasein@ne{quint}%
\easing@pgfmathinstall{quint}%
%    \end{macrocode}
% \end{macro}
% \end{macro}
% \end{macro}
%
% \end{macro}
% \end{macro}
% \end{macro}
%
% \end{macro}
% \end{macro}
% \end{macro}
%
% \end{macro}
% \end{macro}
% \end{macro}
%
% \begin{macro}{\easing@backstep@ne}
% \begin{macro}{\easing@backeasein@ne}
% \begin{macro}{\easing@backeaseout@ne}
%
% The |back| shape.
%
%    \begin{macrocode}
\pgfkeys{easing,
  back/overshoot/.estore in=\easing@param@back@overshoot,
  back/overshoot/.default=1.6,
  back/overshoot}%
\def\easing@backeasein@ne#1{%
  \begingroup
  \pgf@x#1pt
  \edef\pgf@temp{\pgfmath@tonumber\pgf@x}%
  \advance\pgf@x -1pt
  \pgf@x\easing@param@back@overshoot\pgf@x
  \advance\pgf@x\pgf@temp pt
  \pgf@x\pgf@temp\pgf@x
  \pgf@x\pgf@temp\pgf@x
  \pgfmathreturn\pgf@x
  \endgroup
}%
\easing@derive@step@nefromeasein@ne{back}%
\easing@derive@easeout@nefromeasein@ne{back}%
\easing@pgfmathinstall{back}%
%    \end{macrocode}
% \end{macro}
% \end{macro}
% \end{macro}
%
% \begin{macro}{\easing@sinestep@ne}
% \begin{macro}{\easing@sineeasein@ne}
% \begin{macro}{\easing@sineeaseout@ne}
%
% The |sine| shape.
%
% We write down both the |easein| and |step| forms of this, since they
% are simple compared to what would have been obtained by
% |\easing@derive|--.
%
%    \begin{macrocode}
\def\easing@sineeasein@ne#1{%
  \begingroup
  \pgf@x#1pt
  \multiply\pgf@x 90
  \easing@cos{\pgfmath@tonumber\pgf@x}%
  \pgf@x\pgfmathresult pt
  \multiply\pgf@x -1
  \advance\pgf@x 1pt
  \pgfmathreturn\pgf@x
  \endgroup
}%
\def\easing@sinestep@ne#1{%
  \begingroup
  \pgf@x#1pt
  \multiply\pgf@x 180
  \easing@cos{\pgfmath@tonumber\pgf@x}%
  \pgf@x\pgfmathresult pt
  \divide\pgf@x 2
  \multiply\pgf@x -1
  \advance\pgf@x 0.5pt
  \pgfmathreturn\pgf@x
  \endgroup
}%
\easing@derive@easeout@nefromeasein@ne{sine}%
\easing@pgfmathinstall{sine}%
%    \end{macrocode}
% \end{macro}
% \end{macro}
% \end{macro}
%
% \begin{macro}{\easing@expstep@ne}
% \begin{macro}{\easing@expeasein@ne}
% \begin{macro}{\easing@expeaseout@ne}
%
% The |exp| shape.
%
%    \begin{macrocode}
\pgfkeys{easing,
  exp/speed/.estore in=\easing@param@exponent@speed,
  exp/speed/.default=7.2,
  exp/speed}%
\def\easing@expeasein@ne#1{%
  \begingroup
  \pgf@x#1pt
  \advance\pgf@x -1pt
  \pgf@x\easing@param@exponent@speed\pgf@x
  \easing@exp{\pgfmath@tonumber\pgf@x}%
  \pgfmathsmuggle\pgfmathresult
  \endgroup
}%
\easing@derive@step@nefromeasein@ne{exp}%
\easing@derive@easeout@nefromeasein@ne{exp}%
\easing@pgfmathinstall{exp}%
%    \end{macrocode}
%
% \end{macro}
% \end{macro}
% \end{macro}
%
% \begin{macro}{\easing@circstep@ne}
% \begin{macro}{\easing@circeasein@ne}
% \begin{macro}{\easing@circeaseout@ne}
%
% The |circ| shape.
%
%    \begin{macrocode}
\def\easing@circeasein@ne#1{%
  \begingroup
  \pgf@x#1pt
  \advance\pgf@x -1pt
  \edef\pgf@temp{\pgfmath@tonumber\pgf@x}%
  \pgf@x\pgf@temp\pgf@x
  \multiply\pgf@x -1
  \advance\pgf@x 1pt
  \easing@sqrt{\pgfmath@tonumber\pgf@x}%
  \pgfmathsmuggle\pgfmathresult
  \endgroup
}%
\easing@derive@step@nefromeasein@ne{circ}%
\easing@derive@easeout@nefromeasein@ne{circ}%
\easing@pgfmathinstall{circ}%
%    \end{macrocode}
%
% \end{macro}
% \end{macro}
% \end{macro}
%
% \begin{macro}{\easing@elasticstep@ne}
% \begin{macro}{\easing@elasticeasein@ne}
% \begin{macro}{\easing@elasticeaseout@ne}
%
% The |elastic| shape.
%
%    \begin{macrocode}
\pgfkeys{easing,
  elastic/frequency/.estore in=\easing@param@elastic@frequency,
  elastic/damping/.estore in=\easing@param@elastic@damping,
  elastic/frequency/.default=3,
  elastic/damping/.default=7.2,
  elastic/frequency, elastic/damping}%
\def\easing@elasticeasein@ne#1{%
  \begingroup
  \pgf@xa#1pt
  \advance\pgf@xa -1pt
  \pgf@xb-\pgf@xa
  \pgf@xa\easing@param@elastic@damping\pgf@xa
  \easing@exp{\pgfmath@tonumber\pgf@xa}%
  \pgf@xa\pgfmathresult pt
  \pgf@xb 360\pgf@xb
  \pgf@xb\easing@param@elastic@frequency\pgf@xb
  \easing@cos{\pgfmath@tonumber\pgf@xb}%
  \pgf@xa\pgfmathresult\pgf@xa
  \pgfmathreturn\pgf@xa
  \endgroup
}%
\easing@derive@step@nefromeasein@ne{elastic}%
\easing@derive@easeout@nefromeasein@ne{elastic}%
\easing@pgfmathinstall{elastic}%
%    \end{macrocode}
% \end{macro}
% \end{macro}
% \end{macro}
%
% \Finale
